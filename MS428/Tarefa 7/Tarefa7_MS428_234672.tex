\documentclass[a4paper, 12pt]{article}

\usepackage[portuguese]{babel}
\usepackage{blindtext}
\usepackage{mathptmx}
\usepackage{microtype}
\usepackage{enumitem}
\usepackage{amsmath}
\usepackage{index}
\usepackage{fancyhdr}
\usepackage{tikz}
\usepackage{amssymb}
\usepackage{float}
\usepackage{nicematrix}
\usepackage{xcolor}
\usepackage{soulutf8}
\usepackage[hyphens]{url}
\usepackage{bm}
\usetikzlibrary{matrix}
\usetikzlibrary{patterns,decorations.pathreplacing}
\usepackage{graphicx}
\graphicspath{ {./} }

\def\checkmark{\tikz\fill[scale=0.4](0,.35) -- (.25,0) -- (1,.7) -- (.25,.15) -- cycle;} 
\DeclareUnicodeCharacter{2212}{-}
\newcommand*\circled[1]{\tikz[baseline=(char.base)]{
            \node[shape=circle,draw,inner sep=2pt] (char) {#1};}}
\newcommand*\R{\mathbb{R}}
\newcolumntype{C}{>{\centering\let\newline\\\arraybackslash\ $}m{3cm}<{$}}


\begin{document}
\title{\Large{\textbf{Tarefa 6}}}
\author{
\begin{tabular}{c r}
Gabriel Belém Barbosa&RA: 234672
\end{tabular}
}
\date{21 de Novembro de 2021}

\maketitle
\let\cleardoublepage\clearpage
\newpage
\setcounter{page}{2}
\tableofcontents
\newpage
\section{Dual}
Colocando o PL na forma padrão, tem-se
$$
\begin{aligned}
&\min f(\mathbf{x})=x_{1}+12 x_{2}+3 x_{3}+4 x_{4} \\
&\text { Sujeito a: }\left\{\begin{array}{l}
x_{1}+3 x_{2}-2 x_{3}-2 x_{4}-x_{5}=2 \\
-x_{1}+2 x_{2}-3 x_{3}+3 x_{4}-x_{6}=1 \\
x_{1, \ldots, 6} \geq 0
\end{array}\right.
\end{aligned}
$$
Cujo dual é
$$
\begin{aligned}
\max g(\lambda)=& 2 \lambda_{1}+\lambda_{2} \\
\text { Sujeito a: } &\left\{\begin{array}{l}
\lambda_{1}-\lambda_{2} \leq 1 \\
3 \lambda_{1}+2 \lambda_{2} \leq 12 \\
-2 \lambda_{1}-3 \lambda_{2} \leq 3 \\
-2 \lambda_{1}+3 \lambda_{2} \leq 4 \\
-\lambda_{1} \leq 0 \Rightarrow \lambda_{1} \geq 0 \\
-\lambda_{2} \leq 0 \Rightarrow \lambda_{2} \geq 0
\end{array}\right.
\end{aligned}
$$
Para a primeira iteração, $\lambda=(0,0)^{T}$ é claramente uma solução factível para o dual, isto é
$$
B=\left(\begin{array}{ll}
a_{5} & a_{6}
\end{array}\right)=\left(\begin{array}{cc}
-1 & 0 \\
0 & -1
\end{array}\right)
$$
\section{Método dual simplex}
\begin{itemize}
\item Primeira iteração
\begin{itemize}
\item Passo 1: Cálculo da solução básica dual e custos relativos
$$
\begin{gathered}
\underbrace{\left(\begin{array}{cc}
-1 & 0 \\
0 & -1
\end{array}\right)}_{B^{T}} \underbrace{\left(\begin{array}{l}
\hat{\lambda}_{B_{1}}\left(\lambda_{1}\right) \\
\hat{\lambda}_{B_{2}}\left(\lambda_{2}\right)
\end{array}\right)}_{\hat{\lambda}_{\mathrm{B}}}=\underbrace{\left(\begin{array}{l}
0 \\
0
\end{array}\right)}_{\mathrm{c}_{\mathrm{B}}} \\
\Rightarrow \hat{\lambda}_{\mathrm{B}}=B^{T} \backslash \mathbf{c}_{\mathrm{B}}=\left(\begin{array}{l}
0 \\
0
\end{array}\right) \geq 0
\end{gathered}
$$
Solução factível; cálculo dos custos relaticos: 

Associado a $x_{1}$ :
$$
\hat{c}_{N_{1}}=c_{N_{1}}-\hat{\lambda}_{\mathbf{B}}^{T} \mathbf{a}_{\mathbf{N}_{1}}=1
$$
Associado a $x_{2}$ :
$$
\hat{c}_{N_{2}}=c_{N_{2}}-\hat{\lambda}_{\mathbf{B}}^{T} \mathbf{a}_{\mathbf{N}_{2}}=12
$$
Associado a $x_{3}$ :
$$
\hat{c}_{N_{3}}=c_{N_{3}}-\hat{\lambda}_{\mathbf{B}}^{T} \mathbf{a}_{\mathbf{N}_{3}}=3
$$
Associado a $x_{4}$ :
$$
\hat{c}_{N_{3}}=c_{N_{3}}-\hat{\lambda}_{\mathbf{B}}^{T} \mathbf{a}_{\mathbf{N}_{3}}=4
$$
\item Passo 2: Teste de otimalidade
\begin{itemize}
\item 2.1: Cálculo da solução básica primal
\[
\underbrace{
\begin{pmatrix}
-1&0\\
0&-1
\end{pmatrix}}_B
\underbrace{
\begin{pmatrix}
\hat{x}_{B_1} (x_5)\\
\hat{x}_{B_2} (x_6)
\end{pmatrix}}_{\mathbf{\hat{x}_B}}
=
\underbrace{
\begin{pmatrix}
2\\
1
\end{pmatrix}}_{b}
\]
\[
\Rightarrow \mathbf{\hat{x}_B} = B \setminus b =
\begin{pmatrix}
-2\\
-1
\end{pmatrix}
\ngeq 0
\]
\item 2.2: Custos relativos\\
Associado a $x_1$:
\[
\hat{c}_{N_1}=c_{N_1}-\lambda^T\mathbf{a_{N_1}}=-3-(0,1000)(1,2)^T=-2003
\]
Associado a $x_2$:
\[
\hat{c}_{N_2}=c_{N_2}-\lambda^T\mathbf{a_{N_2}}=4-(0,1000)(1,3)^T=-2996
\]
Associado a $x_4$:
\[
\hat{c}_{N_3}=c_{N_3}-\lambda^T\mathbf{a_{N_3}}=-(0,1000)(0,-1)^T=1000
\]

\item 2.3: Escolha da variável a entrar na base
\[
min\{ \hat{c}_{N_j}, j=1,2\}=min\{-2003, -2996, 1000\}=-2996
\]
Logo $\hat{x}_{N_2} (x_2)$ entrará na base. 
		\end{itemize}
		\item Passo 3: Teste de otimalidade
\[
min\{ \hat{c}_{N_j}, j=1,2\}=min\{-2003, -2996, 1000\}=-2996<0
\]
A solução básica não é ótima, logo o método prossegue.
		\item Passo 4: Cálculo da direção simplex
\[
\mathbf{y}=B^{-1}\mathbf{a_{N_2}}
\]
\[
\Rightarrow B \mathbf{y}=\mathbf{a_{N_2}}=(1, 3)^T 
\]
\[
\Rightarrow \mathbf{y}= B\setminus(1, 3)^T =(1, 3)^T 
\]
		\item Passo 5: Determinar passo e variável a sair da base
\[
\mathbf{y}\nleq 0
\]
Logo o método prossegue. Determinar a variável a sair da base
\begin{align}
\hat{\epsilon}&=min\left\{
\frac{\hat{x}_{B_i}}{y_i},\text{ t.q. }y_i>0, j=1,2,3
\right\} \nonumber \\
&=min\left\{
\frac{\hat{x}_{B_1}}{y_1}, \frac{\hat{x}_{B_2}}{y_2}
\right\} \nonumber \\
&=min \left\{
\frac{4}{1}, \frac{18}{3}
\right\}
=4 \nonumber
\end{align}
Logo  $\hat{x}_{B_1} (x_3)$ deve sair da base.
		\item Passo 6: Trocar a primeira coluna de $B$ pela segunda coluna de $N$.
	\end{itemize}
	\item Segunda iteração
	
A solução ótima do PL é $\mathbf{x*}=(0,4,0,0,6)^T$, e o valor ótimo da função auxiliar é $f(\mathbf{x*})=6016$ (para a função da forma padrão, o valor é $g(x*)=\mathbf{c^Tx*}=(-3,4,0,0)(0,4,0,0)^T=16$, e no PL original $z*=-g(x*)=-16$).
\subsection{Análise de solução}
Como a variável auxiliar $z_1$ pertence à base ao fim do processo, o PL inicial não tem solução factível, e portanto não tem solução, como visto em aula.
\end{document}
