\documentclass[a4paper, 12pt]{article}

\usepackage[portuguese]{babel}
\usepackage{blindtext}
\usepackage{mathptmx}
\usepackage{microtype}
\usepackage{enumitem}
\usepackage{amsmath}
\usepackage{index}
\usepackage{fancyhdr}
\usepackage{tikz}
\usepackage{amssymb}
\usepackage{float}
\usepackage{nicematrix}
\usepackage{xcolor}
\usepackage{soulutf8}
\usepackage[hyphens]{url}
\usetikzlibrary{matrix}
\usetikzlibrary{patterns,decorations.pathreplacing}
\usepackage{graphicx}
\graphicspath{ {./} }

\DeclareUnicodeCharacter{2212}{-}
\newcommand*\circled[1]{\tikz[baseline=(char.base)]{
            \node[shape=circle,draw,inner sep=2pt] (char) {#1};}}
\newcommand*\R{\mathbb{R}}
\newcolumntype{C}{>{\centering\let\newline\\\arraybackslash\ $}m{0.12cm}<{$}}


\begin{document}
\title{\Large{\textbf{Tarefa 3}}}
\author{
\begin{tabular}{c r}
Gabriel Belém Barbosa&RA: 234672
\end{tabular}
}
\date{31 de Agosto de 2021}

\maketitle
\let\cleardoublepage\clearpage
\newpage
\setcounter{page}{2}
\tableofcontents
\newpage

\section{Desenvolvimento}
Sabendo que  se um problema de otimização linear tem uma solução ótima, então existe um vértice ótimo (teorema mostrado em aula), o algoritmo testa todos os vértices (intersecção de pares de limitantes de restrições) aplicando a função objetivo (que se quer minimizar ou maximizar) e, se o valor do vértice atual é maior/menor que o anterior máximo/mínimo, o algoritmo checa se o vértice em questão obedece todas as restrições, e, em caso afirmativo, ele se torna o novo máximo/mínimo. Nas figuras abaixo, a região de factibildiade está colorida em vermelho, a solução ótima está denotada, caso exista, assim como o valor da função objetivo na solução. O vetor $c$ é o vetor gradiente da função, e a curva de nível está posicionada sobre o ponto solução ótima (em caso de segmento solução, a curva de nível estaria sobreposta ao segmento solução), quando existe solução ótima.
\section{Item (a)}
\begin{figure}[H]
    \centering
    \caption{\label{fig:1} Item (a)}
    \includegraphics[width=11cm]{item_a}
\end{figure}
Solução única é $x*=\frac{1}{7}(8,4)$.
\section{Item (b)}
\begin{figure}[H]
    \centering
    \caption{\label{fig:2} Item (b)}
    \includegraphics[width=11cm]{item_b}
\end{figure}
Região de factibilidade vazia: problema sem solução. Curva de nível posicionada junta ao vetor C.
\section{Item (c)}
\begin{figure}[H]
    \centering
    \caption{\label{fig:3} Item (c)}
    \includegraphics[width=11cm]{item_c}
\end{figure}
Região de factibilidade ilimitada na direção de interesse (\textbf{$c^tx\rightarrow \infty$}, e busca-se o máximo). Não existe solução ótima finita. Curva de nível posicionada junta ao vetor C.
\section{Item (d)}
\begin{figure}[H]
    \centering
    \caption{\label{fig:4} Item (d)}
    \includegraphics[width=11cm]{item_d}
\end{figure}
Segmento de reta pontilhado é a solução (infinitas soluções, conjunto limitado de soluções):  $x*=(1-\lambda)(2,0)+\frac{\lambda}{19}(20,45)$, $\lambda \in [0,1]$.
\section{Item (e)}
\begin{figure}[H]
    \centering
    \caption{\label{fig:5} Item (e)}
    \includegraphics[width=11cm]{item_e}
\end{figure}
Solução única é $x*=(5,5)$. Vale ressaltar que o ponto é degenerado, sendo intersecção de $x_1 + x_2 =10$, $x_1 + 2x_2 = 15$ e $2x_1 + x_2 = 15$.
\section{Algoritmo}
A seguir está o algoritmo (em \verb+Octave+) usado no Item (e).
\begin{verbatim}
f=@(x,y) 3*x+5*y;
grad=[3, 5];
agrad=[1,-grad(1)/grad(2)];
restri=[1,1;1,2;2,1;1,-2;1,0;0,1]
res=[10;15;15;1;0;0]
sines=[-1,-1,-1,-1,1,1]
corners=[]
a=[]
min=[0;0;-10000]; #valor inicial dummy
hf=figure;
xlabel ("x1");
ylabel ("x2");
hold on
quiver(0,0,grad(1)/10,grad(2)/10);
for i=1:rows(restri)
    for j=i+1:rows(restri)
        a=[restri(i,:);restri(j,:)]\[res(i);res(j)];
        corners=[corners;transpose(a)];
        if (f(a(1),a(2))>=min(3))
            key=1;
            for p=1:rows(restri)
                if ((restri(p,:)*a>res(p) && sines(p)==-1) || 
                (restri(p,:)*a<res(p) && sines(p)==1))
                    key=0;
                    break;
                end
            endfor
            if key
                min=[a;f(a(1),a(2))];
            end
        end
    endfor
    if (restri(i,2)!=0)
        fplot(@(x) (res(i)-restri(i,1)*x)/restri(i,2),[0,6.5]);
    else
        [restri(i,1),restri(i,2),res(i)]
        if (res(i)==0)
            line([0.01,0.01],[0,20]); 
	#altura escolhida à mão de acordo com o gráfico
        else
            line([res(i),res(i)],[0,20]); 
	#altura escolhida à mão de acordo com o gráfico
        end
    end
endfor
text(0.35,0.55,'C','FontSize', 14);
scatter(min(1),min(2), "filled");
text(min(1)-0.5,min(2)-1,strcat('x*(f(x*)=',
mat2str(min(3)),')'),'FontSize', 14);
line([min(1)-0.9*agrad(1),min(1)+0.9*agrad(1)],
[min(2)-0.9*agrad(2),min(2)+0.9*agrad(2)]);
legend("hide");
hold off
\end{verbatim}







\end{document}