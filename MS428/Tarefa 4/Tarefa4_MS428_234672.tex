\documentclass[a4paper, 12pt]{article}

\usepackage[portuguese]{babel}
\usepackage{blindtext}
\usepackage{mathptmx}
\usepackage{microtype}
\usepackage{enumitem}
\usepackage{amsmath}
\usepackage{index}
\usepackage{fancyhdr}
\usepackage{tikz}
\usepackage{amssymb}
\usepackage{float}
\usepackage{nicematrix}
\usepackage{xcolor}
\usepackage{soulutf8}
\usepackage[hyphens]{url}
\usetikzlibrary{matrix}
\usetikzlibrary{patterns,decorations.pathreplacing}
\usepackage{graphicx}
\graphicspath{ {./} }

\def\checkmark{\tikz\fill[scale=0.4](0,.35) -- (.25,0) -- (1,.7) -- (.25,.15) -- cycle;} 
\DeclareUnicodeCharacter{2212}{-}
\newcommand*\circled[1]{\tikz[baseline=(char.base)]{
            \node[shape=circle,draw,inner sep=2pt] (char) {#1};}}
\newcommand*\R{\mathbb{R}}
\newcolumntype{C}{>{\centering\let\newline\\\arraybackslash\ $}m{3cm}<{$}}


\begin{document}
\title{\Large{\textbf{Tarefa 4}}}
\author{
\begin{tabular}{c r}
Gabriel Belém Barbosa&RA: 234672
\end{tabular}
}
\date{31 de Agosto de 2021}

\maketitle
\let\cleardoublepage\clearpage
\newpage
\setcounter{page}{2}
\tableofcontents
\newpage

\section{Exercício 1}
\subsection{Item (a)}
$max\text{ }z=x_1$
\[
\text{Sujeito a:}\left\{
\begin{array}{l}
x_1+2x_2+x_3=6\\
x_1-x_2+x_4=4\\
x_2+x_5=2\\
x_1\geq 0, x_2\geq 0, x_3\geq 0, x_4\geq 0, x_5\geq 0
\end{array}
\right.
\]
\subsection{Item (b)}
\begin{table}[h]
\centering
\begin{figure}[H]
    \centering
    \caption{\label{fig:4} Região de factibilidade}
    \includegraphics[width=11cm]{b}
\end{figure}
\small
Região de factibilidade em vermelho.
\end{table}
\subsection{Item (c)}
\begin{table}[H]
\centering
\caption{\label{tab:mse} Pontos extremos}
\begin{tabular}{|c||C|C|C|c|}
\hline Nome&\text{Var. não básicas}&\text{Var. básicas}&(x_1, x_2, x_3, x_4, x_5)&Factibilidade\\
\hline \hline X1 &x_1, x_2& x_3, x_4, x_5&(0,0,6,4,2)&\checkmark\\
\hline X2&x_1, x_3& x_2, x_4, x_5&(0,3,0,7,-1)&$\times$\\
\hline X3&x_1, x_4& x_2, x_3, x_5&(0,-4,14,0,6)&$\times$\\
\hline X4& x_1, x_5& x_2, x_3, x_4&(0,2,2,6,0)&\checkmark\\
\hline X5&x_2, x_3& x_1, x_4, x_5&(6,0,0,-2,2)&$\times$\\
\hline X6&x_2, x_4&x_1, x_3, x_5&(4,0,2,0,2)&\checkmark\\
\hline X7&x_2, x_5&x_1, x_3, x_4&\text{---}&---\\
\hline X8& x_3, x_4& x_1, x_2, x_5&\frac{1}{3}(14,2,0,0,4)&\checkmark\\
\hline X9&x_3, x_5& x_1, x_2, x_4&(2,2,0,4,0)&\checkmark\\
\hline X10&x_4, x_5& x_1, x_2, x_3&(6,2,-4,0,0)&$\times$\\ \hline
\end{tabular}
\end{table}
A tabela acima mostra os pontos extremos encontrados computacionalmente (algoritmo apresentado ao final do trabalho na seção Algoritmo), sendo que a factibilidade foi definida através do cumprimento (ou não, no caso de infactíveis) de todas as condições de positividade. Repare que o sistema básico associado ao que seria o ponto X7 não tem solução, logo foi identificado com um ---.
\begin{figure}[H]
    \centering
    \caption{\label{fig:4} Pontos extremos no plano}
    \includegraphics[width=11cm]{c}
\end{figure}
\subsection{Item (d)}
Como visto na Tabela 1, $x_2$ entra e $x_3$ sai das variáveis básicas no movimento de $(4,0,2,0,2)$ a $\frac{1}{3}(14,2,0,0,4)$.
\section{Exercício 2}
\subsection{Item (a)}
O sistema em questão é
\[
\underbrace{
\begin{pmatrix}
1&-2&-4\\
2&-4&2\\
2&-5&3
\end{pmatrix}}_B
\underbrace{
\begin{pmatrix}
x_1\\
x_2\\
x_5
\end{pmatrix}}_{\hat{x}_B}
=
\underbrace{
\begin{pmatrix}
2\\
-6\\
-7
\end{pmatrix}}_{b}
\]
No qual a matriz $B$ tem $det(B)=10\neq0$, logo forma uma base. A solução básica em questão é $x*=(-2,0,0,0,-1)$ (obtida computacionalmente através do \verb+Octave+). Esse ponto desobedece a condição de positividade das variáveis $x_1$ e $x_5$, logo não é uma solução básica factível. Em adendo, como $x_2=0$ e $x_2$ é uma variável da base, temos que essa solução é degenerada.
\section{Algoritmo}
O seguinte algoritmo para o Exercício 1 foi implementado em \verb+Octave+.
\begin{verbatim}
restri=[1,2,1,0,0;1,-1,0,1,0;0,1,0,0,1]
res=[6;4;2;0;0]
k=0;
solsol=[];
hf=figure;
xlabel ("x1");
ylabel ("x2");
hold on
for i=1:size(restri,2)
    for j=i+1:size(restri,2)
        k++;
        aux=restri;
        aux(:,[i,j])=[];
        sol=aux\[res(1);res(2);res(3)];
        if (aux*sol==[res(1);res(2);res(3)])
            solsol=[solsol,[sol;i;j]]
            if (i==1)
                if (j==2)
                    text(0.1, 0.1, strcat("X",num2str(k)))
                    scatter(0, 0, "filled")
                else
                    text(0.1, sol(1)+0.1, strcat("X",num2str(k)))
                    scatter(0, sol(1), "filled")
                end
            elseif (i==2)
                text(sol(1)+0.1, 0.1, strcat("X",num2str(k)))
                scatter(sol(1), 0, "filled")
            else
                text(sol(1)+0.1, sol(2)+0.1, strcat("X",num2str(k)))
                scatter(sol(1), sol(2), "filled")
            end
         end
    endfor
endfor
for i=1:size(restri,1)
    if (restri(i,2)!=0)
        fplot(@(x) (res(i)-restri(i,1)*x)/restri(i,2),[0,6.5]);
    end
endfor
line([0.01,0.01],[-4,6]);
line([0,6.5],[0,0]);
legend("hide");
print (hf, "c.png");
hold off
\end{verbatim}
\end{document}