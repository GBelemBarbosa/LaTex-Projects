\documentclass[a4paper, 12pt]{article}

\usepackage[portuguese]{babel}
\usepackage{blindtext}
\usepackage{mathptmx}
\usepackage{microtype}
\usepackage{enumitem}
\usepackage{amsmath}
\usepackage{index}
\usepackage{fancyhdr}
\usepackage{tikz}
\usepackage{amssymb}
\usepackage{float}
\usepackage{nicematrix}
\usepackage{xcolor}
\usepackage{soulutf8}
\usepackage[hyphens]{url}
\usepackage{bm}
\usetikzlibrary{matrix}
\usetikzlibrary{patterns,decorations.pathreplacing}
\usepackage{graphicx}
\graphicspath{ {./} }

\def\checkmark{\tikz\fill[scale=0.4](0,.35) -- (.25,0) -- (1,.7) -- (.25,.15) -- cycle;} 
\DeclareUnicodeCharacter{2212}{-}
\newcommand*\circled[1]{\tikz[baseline=(char.base)]{
            \node[shape=circle,draw,inner sep=2pt] (char) {#1};}}
\newcommand*\R{\mathbb{R}}
\newcolumntype{C}{>{\centering\let\newline\\\arraybackslash\ $}m{3cm}<{$}}


\begin{document}
\title{\Large{\textbf{Tarefa 9}}}
\author{
\begin{tabular}{c r}
Gabriel Belém Barbosa&RA: 234672
\end{tabular}
}
\date{01 de Outubro de 2021}

\maketitle
\let\cleardoublepage\clearpage
\newpage
\setcounter{page}{2}
\tableofcontents
\newpage
\section{Exercício 1}
\subsection{Item (a)}
$max\text{ }g(x)=6p_1+4p_2$
\[
\text{Sujeito a:}\left\{
\begin{array}{l}
4p_1+p_2\leq2\\
3p_1+2p_2\leq3\\
-p_1+5p_2\leq1\\
p_1\geq 0, p_2\geq 0
\end{array}
\right.
\]
\subsection{Item (b)}
$min\text{ }g(x)=12p_1+4p_2$
\[
\text{Sujeito a:}\left\{
\begin{array}{l}
7p_1+p_2\geq10\\
3p_1+5p_2\geq-2\\
p_1, p_2\text{ livre}
\end{array}
\right.
\]
\subsection{Item (c)}
$min\text{ }g(x)=3p_1+8p_2+2p_3$
\[
\text{Sujeito a:}\left\{
\begin{array}{l}
p_1+4p_2+p_3\geq3\\
-p_1+p_2+3p_3\geq1\\
p_1\leq0,  p_2\leq0, p_3\geq0
\end{array}
\right.
\]
\subsection{Item (d)}
$max\text{ }g(x)=2p_1+9p_2$
\[
\text{Sujeito a:}\left\{
\begin{array}{l}
2p_1+5p_2\geq7\\
-p_1+6p_2\geq0.5\\
2p_1+p_2=3\\
p_1\leq 0, p_2\geq 0
\end{array}
\right.
\]
\section{Exercício 2}
\subsection{Item (a)}
O dual do problema apresentado é\\
$max\text{ }-c^Tp$
\[
\text{Sujeito a:}\left\{
\begin{array}{l}
A^Tp\leq c\\
p\geq 0
\end{array}
\right.
\]
Transformando o problema de máximo em um de mínimo tem-se o objetivo $min\text{ }c^Tp$. Pela antissimetria de $A$
\[
A^Tp=-Ap\leq c
\]
\[
\Rightarrow Ap\geq c
\]
Logo o dual pode ser escrito como
$min\text{ }c^Tp$
\[
\text{Sujeito a:}\left\{
\begin{array}{l}
Ap\geq c\\
p\geq 0
\end{array}
\right.
\]
Que é exatamente igual ao primal.
\subsection{Item (b)}
Pelo teorema forte da dualidade, supondo por contradição que um PL factível desse tipo possui solução ilimitada, logo seu dual não possui solução factível. Como, pelo item (a), o primal é igual ao dual nesse tipo de PL, cai-se em contradição pela hipótese de que o PL tinha solução factível. Portanto, se um PL desse tipo tem solução, esta deve ser limitada.
\section{Exercício 3}
\subsection{Item (a)}
Dual:\\
$max\text{ }g(x)=2p_1+p_2$
\[
\text{Sujeito a:}\left\{
\begin{array}{l}
2p_1+3p_2\geq6\\
3p_1-p_2\leq9\\
3p_1+4p_2\leq24\\
p_1\geq 0, p_2\geq 0
\end{array}
\right.
\]
Como deseja-se maximizar a função objetivo, é necessário viajar na direção do gradiente. A curva de nível da função objetivo intersecta a região de factibilidade por último em $\mathbf{x*}=(4,3)^T$, como pode ser visto no gráfico abaixo, cujo valor ótimo é, da função objetivo, $f(\mathbf{x*})=2\cdot 4+3=11$.
\begin{table}[H]
\centering
\begin{figure}[H]
    \centering
    \caption{\label{fig:1} Resolução gráfica}
    \includegraphics[width=11cm]{a}
\end{figure}
\small
Região de factibilidade em vermelho, vetor gradiente C e solução ótima x* denotados e curvas de nível pontilhadas.
\end{table}
Pelo teorema forte da dualidade, o primal possui solução ótima. Pelo teorema das folgas complementares
\[
(x_1^*,x_2^*,x_3^*)
\begin{pmatrix}
6-17\\
9-9\\
24-24
\end{pmatrix}
=
\begin{pmatrix}
0\\
0\\
0
\end{pmatrix}
\]
\[
\Rightarrow -11x_1^*=0\Rightarrow x_1^*=0
\]
Sendo
\[
A^Tp^*=
\begin{pmatrix}
17\\
9\\
24
\end{pmatrix}
\]
E
\[
(4,3)
\begin{pmatrix}
2x_1^*+3x_2^*+3x_3^*-2\\
3x_1^*-x_2^*+4x_3^*-1
\end{pmatrix}
=
=
\begin{pmatrix}
0\\
0
\end{pmatrix}
\]
\[
\Rightarrow 4(3x_2^*+3x_3^*-2)=3x_2^*+3x_3^*-2=0 
\]
\[
\Rightarrow 3(4x_3^*-x_2^*-1)=4x_3^*-x_2^*-1=0 
\]
Resolvendo o sitema acima
\[
\Rightarrow 15x_3^*=5
\]
\[
\Rightarrow x_3^*=\frac{1}{3}
\]
\[
\Rightarrow x_2^*=4x_3^*-1=\frac{1}{3}
\]
Logo a solução ótima é $x^*=(0,\frac{1}{3},\frac{1}{3})$, cujo valor é $f(x^*)=6\cdot 0+9\cdot \frac{1}{3}+24\cdot\frac{1}{3}=11$.
\subsection{Item (b)}
Dual:\\
$min\text{ }g(x)=-3p_1-2p_2$
\[
\text{Sujeito a:}\left\{
\begin{array}{l}
3p_1+p_2\geq6\\
p_1+2p_2\geq4\\
-p_1+2p_2\geq-4\\
3p_1-2p_2\geq0\\
p_1\geq 0, p_2\geq 0
\end{array}
\right.
\]
Como deseja-se minimizar a função objetivo, é necessário viajar na direção contrária ao gradiente. A curva de nível da função objetivo semre intersecta a região de factibilidade nessa direção, como pode ser visto no gráfico abaixo. Logo o dual não possui solução ótima limitada.
\begin{table}[H]
\centering
\begin{figure}[H]
    \centering
    \caption{\label{fig:1} Resolução gráfica}
    \includegraphics[width=11cm]{b}
\end{figure}
\small
Região de factibilidade em vermelho, vetor gradiente C e solução ótima x* denotados e curvas de nível pontilhadas.
\end{table}
Pelo teorema forte da dualidade, o primal não possui solução factível.
\subsection{Item (c)}
Dual:\\
$max\text{ }g(x)=2p_1-p_2$
\[
\text{Sujeito a:}\left\{
\begin{array}{l}
2p_1+p_2\leq2\\
-p_1-p_2\leq-1\\
-p_1+3p_2\leq2\\
p_1\geq 0, p_2\geq 0
\end{array}
\right.
\]
Como deseja-se maximizar a função objetivo, é necessário viajar na direção do gradiente. A curva de nível da função objetivo intersecta a região de factibilidade por último em $\mathbf{x*}=(1,0)^T$, como pode ser visto no gráfico abaixo, cujo valor ótimo é, da função objetivo, $f(\mathbf{x*})=2\cdot 1+0=2$.
\begin{table}[H]
\centering
\begin{figure}[H]
    \centering
    \caption{\label{fig:1} Resolução gráfica}
    \includegraphics[width=11cm]{c}
\end{figure}
\small
Região de factibilidade em vermelho, vetor gradiente C e solução ótima x* denotados e curvas de nível pontilhadas.
\end{table}
Pelo teorema forte da dualidade, o primal possui solução ótima. Pelo teorema das folgas complementares
\[
(x_1^*,x_2^*,x_3^*)
\begin{pmatrix}
2-2\\
-1+1\\
2+1
\end{pmatrix}
=
\begin{pmatrix}
0\\
0\\
0
\end{pmatrix}
\]
\[
\Rightarrow 3x_3^*=0\Rightarrow x_3^*=0
\]
Sendo
\[
A^Tp^*=
\begin{pmatrix}
2\\
-1\\
-1
\end{pmatrix}
\]
E
\[
(1,0)
\begin{pmatrix}
2x_1^*-x_2^*-x_3^*-2\\
x_1^*-x_2^*+3x_3^*+1
\end{pmatrix}
=
\begin{pmatrix}
0\\
0
\end{pmatrix}
\]
\[
\Rightarrow 1(2x_1^*-x_2^*-2)=0 
\]
\[
\Rightarrow 2x_1^*=x_2^*+2
\]
Como, pela segunda restrição do primal, substituindo as relações acima
\[
x_1^* − x_2^* + 3x_3^*=-x_1^*+2=-\frac{x_2^*}{2}+1\geq-1
\] 
Tem-se que esse segmento de reta é limitado por $x_1^*=3$ ($x_2^*=4$), e por $x_2^*=0$ ($x_1^*=1$) pela não negatividade. Como a solução ótima, se existir, estará em um extremo, fica fácil conferir que $x^*=(1,0,0)$ é a solução ótima do primal, e o valor ótimo é portanto $f(x^*)=2\cdot 1-0+2\cdot 0=2$.
\end{document}