\documentclass[a4paper, 12pt]{article}

\usepackage[portuguese]{babel}
\usepackage{blindtext}
\usepackage{mathptmx}
\usepackage{microtype}
\usepackage{enumitem}
\usepackage{amsmath}
\usepackage{index}
\usepackage{fancyhdr}
\usepackage{tikz}
\usepackage{amssymb}
\usepackage{float}
\usepackage{nicematrix}
\usepackage{xcolor}
\usepackage{soulutf8}
\usepackage[hyphens]{url}
\usepackage{bm}
\usetikzlibrary{matrix}
\usetikzlibrary{patterns,decorations.pathreplacing}
\usepackage{graphicx}
\graphicspath{ {./} }

\def\checkmark{\tikz\fill[scale=0.4](0,.35) -- (.25,0) -- (1,.7) -- (.25,.15) -- cycle;} 
\DeclareUnicodeCharacter{2212}{-}
\newcommand*\circled[1]{\tikz[baseline=(char.base)]{
            \node[shape=circle,draw,inner sep=2pt] (char) {#1};}}
\newcommand*\R{\mathbb{R}}
\newcolumntype{C}{>{\centering\let\newline\\\arraybackslash\ $}m{3cm}<{$}}


\begin{document}
\title{\Large{\textbf{Tarefa 5}}}
\author{
\begin{tabular}{c r}
Gabriel Belém Barbosa&RA: 234672
\end{tabular}
}
\date{13 de Setembro de 2021}

\maketitle
\let\cleardoublepage\clearpage
\newpage
\setcounter{page}{2}
\tableofcontents
\newpage

\section{Exercício 1}
\subsection{Item (a)}
O sistema em questão é
\[
\underbrace{
\begin{pmatrix}
1&0&0\\
0&1&0\\
0&0&1
\end{pmatrix}}_B
\underbrace{
\begin{pmatrix}
\mathbf{x_3}\\
\mathbf{x_4}\\
\mathbf{x_5}
\end{pmatrix}}_{\hat{x}_B}
=
\underbrace{
\begin{pmatrix}
0\\
4\\
15
\end{pmatrix}}_{b}
\]
Cuja solução é $\hat{x}_B=(0, 4, 15)^T$. Essa solução é factível, uma vez que respeita a não negatividade das variáveis da base (as variáveis fora da base são obviamente nulas, e portanto não negativas também). Para testar a optimalidadade da solução, como $c^T_B$, vetor com os coeficientes das variáveis básicas na função objetivo, é nulo, tem-se que o vetor multiplicador simplex $\lambda^T=c^T_BB^{-1}=\vec{0}$. Dos slides da aula e dessa nulidade
\[
f(x)=f(\hat{x})+(c_{N_1}-\lambda^T a_{N_1})x_1+(c_{N_2}-\lambda^T a_{N_2})x_2=c_{N_1}x_1+c_{N_2}x_2
\]
Onde foi usado que $f(\hat{x})=-\mathbf{x_1}-3\mathbf{x_2}=-0-3\cdot0=0$. Agora, sendo da função minimizadora $c_{N_1}=-1$ e  $c_{N_2}=-3$
\[
f(x)=-x_1-3x_2
\]
Logo, visto que o objetivo é encontrar o minimizador, qualquer uma das variáveis fora da base poderia entrar nela para tal, pois os seus respectivos coeficientes são negativos, e a solução não é ótima ($\exists \hat{c}_{N_j}=c_{N_j}-\lambda^T a_{N_j}<0$, sendo $\hat{c}_{N_j}$ o custo relativo da variável $j$ fora da base).
\subsection{Item (b)}
Colocando o PL na forma padrão, tem-se
\\$min\text{ }z=-3x_1-2x_2-x_3$
\[
\text{Sujeito a:}\left\{
\begin{array}{l}
3x_1-3x_2+2x_3+x_4=3\\
-x_1+2x_2+x_3+x_5=6\\
x_1\geq 0, x_2\geq 0, x_3\geq 0, x_4\geq 0, x_5\geq 0
\end{array}
\right.
\]
O sistema em questão é
\[
\underbrace{
\begin{pmatrix}
3&-3\\
-1&2
\end{pmatrix}}_B
\underbrace{
\begin{pmatrix}
\mathbf{x_1}\\
\mathbf{x_2}
\end{pmatrix}}_{\hat{x}_B}
=
\underbrace{
\begin{pmatrix}
3\\
4
\end{pmatrix}}_{b}
\]
Cuja solução é obviamente $\hat{x}_B=(6, 5)^T$. Essa solução é factível, uma vez que respeita a não negatividade das variáveis da base (as variáveis fora da base são obviamente nulas, e portanto não negativas também). Para testar a optimalidadade da solução, com $c^T_B=(-3, -2)$ da função objetivo e a inversa de $B$ (usando a regra de invertibilidade para matrizes $2\times 2$)
\[
B^{-1}=\frac{1}{3}
\begin{pmatrix}
2&3\\
1&3
\end{pmatrix}
\]
Tem-se que o vetor multiplicador simplex $\lambda^T=c^T_BB^{-1}=(-\frac{8}{3}, -5)$. Dos slides da aula
\[
f(x)=f(\hat{x})+(c_{N_1}-\lambda^T a_{N_1})x_3+f(\hat{x})+(c_{N_2}-\lambda^T a_{N_2})x_4+f(\hat{x})+(c_{N_3}-\lambda^T a_{N_3})x_5
\]
Sendo $a_{N_1}=(2, 1)^T$, $a_{N_2}=(1, 0)^T$ e $a_{N_3}=(0, 1)^T$, dos coeficientes de $x_3$, $x_4$ e $x_5$ nas equações de restrição, respectivamente, e $c_{N_1}=-1$ e $c_{N_2}=c_{N_3}=0$ da função objetivo. Substituindo
\begin{align}
f(x)&=(-1+(\frac{8}{3}, 5)(2, 1)^T)x_3+(\frac{8}{3}, 5)(1, 0)^Tx_4+(\frac{8}{3}, 5)(0, 1)^Tx_5-28 \nonumber \\
&=\frac{28}{3}x_3+\frac{8}{3}x_4+5x_5-28 \nonumber
\end{align}
Onde foi usado que $f(\hat{x})=-3\mathbf{x_1}-2\mathbf{x_2}-\mathbf{x_3}=-3\cdot6-2\cdot5-0=-28$. Logo, visto que o objetivo é encontrar o minimizador, a solução é ótima ($\hat{c}_{N_j}=c_{N_j}-\lambda^T a_{N_j}\geq0$, $j=1,2,3$) e nenhuma variável deve entrar na base.
\end{document}