\documentclass{article}
\usepackage[portuguese]{babel}
\usepackage{blindtext}
\usepackage{mathptmx}
\usepackage{microtype}
\usepackage{enumitem}
\usepackage{amsmath}
\usepackage{index}
\usepackage{fancyhdr}
\usepackage{tikz}
\usepackage{amssymb}
\usepackage{float}
\usepackage{nicematrix}
\usepackage{xcolor}
\usepackage{soulutf8}
\usepackage[hyphens]{url}
\usepackage{bm}
\usetikzlibrary{matrix}
\usetikzlibrary{patterns,decorations.pathreplacing}
\usepackage{graphicx}
\graphicspath{ {./} }

\def\checkmark{\tikz\fill[scale=0.4](0,.35) -- (.25,0) -- (1,.7) -- (.25,.15) -- cycle;} 
\DeclareUnicodeCharacter{2212}{-}
\newcommand*\circled[1]{\tikz[baseline=(char.base)]{
            \node[shape=circle,draw,inner sep=2pt] (char) {#1};}}
\newcommand*\R{\mathbb{R}}
\newcolumntype{C}{>{\centering\let\newline\\\arraybackslash\ $}m{3cm}<{$}}


\begin{document}
\title{\Large{\textbf{Tarefa 10}}}
\author{
\begin{tabular}{c r}
Gabriel Belém Barbosa&RA: 234672
\end{tabular}
}
\date{21 de Novembro de 2021}

\maketitle
\let\cleardoublepage\clearpage
\newpage
\setcounter{page}{2}
\tableofcontents
\newpage
\section{Fase I}
Colocando o PL na forma padrão, tem-se
$$
\begin{aligned}
&\min f(\mathbf{x})=x_{1}+12 x_{2}+3 x_{3}+4 x_{4} \\
&\text {Sujeito a: }\left\{\begin{array}{l}
x_{1}+3 x_{2}-2 x_{3}-2 x_{4}-x_{5}=2 \\
-x_{1}+2 x_{2}-3 x_{3}+3 x_{4}-x_{6}=1 \\
x_{1, \ldots, 6} \geq 0
\end{array}\right.
\end{aligned}
$$
Cujo dual é
$$
\begin{aligned}
\max g(\lambda)=& 2 \lambda_{1}+\lambda_{2} \\
\text {Sujeito a: } &\left\{\begin{array}{l}
\lambda_{1}-\lambda_{2} \leq 1 \\
3 \lambda_{1}+2 \lambda_{2} \leq 12 \\
-2 \lambda_{1}-3 \lambda_{2} \leq 3 \\
-2 \lambda_{1}+3 \lambda_{2} \leq 4 \\
-\lambda_{1} \leq 0 \Rightarrow \lambda_{1} \geq 0 \\
-\lambda_{2} \leq 0 \Rightarrow \lambda_{2} \geq 0
\end{array}\right.
\end{aligned}
$$
Para a primeira iteração, $\lambda=(0,0)^{T}$ é claramente uma solução factível para o dual, isto é
$$
B=\left(\begin{array}{ll}
a_{5} & a_{6}
\end{array}\right)=\left(\begin{array}{cc}
-1 & 0 \\
0 & -1
\end{array}\right)
$$
\section{Fase II}
\begin{itemize}
\item Primeira iteração
\begin{itemize}
\item Passo 1: Cálculo da solução básica dual e custos relativos
$$
\begin{gathered}
\underbrace{\left(\begin{array}{cc}
-1 & 0 \\
0 & -1
\end{array}\right)}_{\mathbf{B}^{T}} \underbrace{\left(\begin{array}{l}
\hat{\lambda}_{B_{1}}\left(\lambda_{1}\right) \\
\hat{\lambda}_{B_{2}}\left(\lambda_{2}\right)
\end{array}\right)}_{\hat{\lambda}_{\mathbf{B}}}=\underbrace{\left(\begin{array}{l}
0 \\
0
\end{array}\right)}_{\mathbf{c}_{\mathbf{B}}} \\
\Rightarrow \hat{\lambda}_{\mathbf{B}}=B^{T} \backslash \mathbf{c}_{\mathbf{B}}=\left(\begin{array}{l}
0 \\
0
\end{array}\right) \geq 0
\end{gathered}
$$
Solução factível; cálculo dos custos relaticos: 

Associado a $x_{1}$ :
$$
\hat{c}_{\mathbf{N}_{1}}=c_{\mathbf{N}_{1}}-\hat{\lambda}_{\mathbf{B}}^{T} \mathbf{a}_{\mathbf{N}_{1}}=1
$$
Associado a $x_{2}$ :
$$
\hat{c}_{\mathbf{N}_{2}}=c_{\mathbf{N}_{2}}-\hat{\lambda}_{\mathbf{B}}^{T} \mathbf{a}_{\mathbf{N}_{2}}=12
$$
Associado a $x_{3}$ :
$$
\hat{c}_{\mathbf{N}_{3}}=c_{\mathbf{N}_{3}}-\hat{\lambda}_{\mathbf{B}}^{T} \mathbf{a}_{\mathbf{N}_{3}}=3
$$
Associado a $x_{4}$ :
$$
\hat{c}_{\mathbf{N}_{4}}=c_{\mathbf{N}_{4}}-\hat{\lambda}_{\mathbf{B}}^{T} \mathbf{a}_{\mathbf{N}_{4}}=4
$$
\item Passo 2: Teste de otimalidade
\begin{itemize}
\item 2.1: Cálculo da solução básica primal
\[
\underbrace{
\begin{pmatrix}
-1&0\\
0&-1
\end{pmatrix}}_{\mathbf{B}}
\underbrace{
\begin{pmatrix}
\hat{x}_{\mathbf{B}_1} (x_5)\\
\hat{x}_{\mathbf{B}_2} (x_6)
\end{pmatrix}}_{\mathbf{\hat{x}_\mathbf{B}}}
=
\underbrace{
\begin{pmatrix}
2\\
1
\end{pmatrix}}_{\mathbf{b}}
\]
\[
\Rightarrow \mathbf{\hat{x}_\mathbf{B}} = \mathbf{B} \setminus \mathbf{b} =
\begin{pmatrix}
-2\\
-1
\end{pmatrix}
\ngeq 0
\]
\item 2.2: Determinar variável a sair da base\\
\[
\hat{x}_{\mathbf{B}_l}=min\{ \hat{x}_{\mathbf{B}_i}, i=1,2\}=min\{-2, -1\}=-2
\]
Como $\hat{x}_\mathbf{B_l}<0$, o método prossegue, e $\hat{x}_\mathbf{B_1}$ sairá da base.
		\end{itemize}
		\item Passo 3: Cálculo da direção dual simplex
\[
\underbrace{
\begin{pmatrix}
-1&0\\
0&-1
\end{pmatrix}}_{\mathbf{B}^T}
\mathbf{\eta}_1
=
\underbrace{
\begin{pmatrix}
-1\\
0
\end{pmatrix}}_{-\mathbf{e}_1}
\]
\[
\Rightarrow \mathbf{\eta}_1 = \mathbf{B} \setminus \mathbf{e}_1 =
\begin{pmatrix}
2\\
0
\end{pmatrix}
\]
		\item Passo 4: Determinar passo e variável a entrar na base
\begin{align*}
\widehat{\delta}&=\frac{\widehat{c}_{\mathbf{N}_k}}{\eta_\ell^T \mathbf{a}_{\mathbf{N}_k}}
=\min_{j=1,...,n-m}\left\{\frac{\widehat{c}_{\mathbf{N}_j}}{\eta_1^T \mathbf{a}_{\mathbf{N}_j}} \text { tal que } \eta_1^T \mathbf{a}_{\mathbf{N}_j}>0\right\}\\
&=\min \left\{
\frac{\widehat{c}_{\mathbf{N}_1}}{\eta_{1}^T \mathbf{a}_{\mathbf{N}_1}}, \frac{\widehat{c}_{\mathbf{N}_2}}{\eta_{1}^T \mathbf{a}_{\mathbf{N}_2}}
\right\}\\
&=\min \left\{
\frac{1}{2}, \frac{12}{6}
\right\}=\frac{1}{2}
\end{align*}
Logo $x_{\mathbf{N}_1}$ ($x_1$) entrará na base.
		\item Passo 5: Nova partição básica

Troca-se a primeira coluna de B pela primeira coluna de N.
	\end{itemize}
	\item Segunda iteração
	\begin{itemize}
\item Passo 1: Cálculo da solução básica dual e custos relativos
$$
\begin{gathered}
\underbrace{\left(\begin{array}{cc}
1 & -1 \\
0 & -1
\end{array}\right)}_{\mathbf{B}^{T}} \underbrace{\left(\begin{array}{l}
\hat{\lambda}_{B_{1}}\left(\lambda_{1}\right) \\
\hat{\lambda}_{B_{2}}\left(\lambda_{2}\right)
\end{array}\right)}_{\hat{\lambda}_{\mathbf{B}}}=\underbrace{\left(\begin{array}{l}
1 \\
0
\end{array}\right)}_{\mathbf{c}_{\mathbf{B}}} \\
\Rightarrow \hat{\lambda}_{\mathbf{B}}=B^{T} \backslash \mathbf{c}_{\mathbf{B}}=\left(\begin{array}{l}
1 \\
0
\end{array}\right) \geq 0
\end{gathered}
$$
Solução factível; cálculo dos custos relaticos: 

Associado a $x_{5}$ :
$$
\hat{c}_{\mathbf{N}_{1}}=c_{\mathbf{N}_{1}}-\hat{\lambda}_{\mathbf{B}}^{T} \mathbf{a}_{\mathbf{N}_{1}}=0-(1,0)
\begin{pmatrix}
-1\\
0
\end{pmatrix}
=1
$$
Associado a $x_{2}$ :
$$
\hat{c}_{\mathbf{N}_{2}}=c_{\mathbf{N}_{2}}-\hat{\lambda}_{\mathbf{B}}^{T} \mathbf{a}_{\mathbf{N}_{2}}=12-(1,0)
\begin{pmatrix}
3\\
2
\end{pmatrix}
=9
$$
Associado a $x_{3}$ :
$$
\hat{c}_{\mathbf{N}_{3}}=c_{\mathbf{N}_{3}}-\hat{\lambda}_{\mathbf{B}}^{T} \mathbf{a}_{\mathbf{N}_{3}}=3-(1,0)
\begin{pmatrix}
-2\\
-3
\end{pmatrix}
=5
$$
Associado a $x_{4}$ :
$$
\hat{c}_{\mathbf{N}_{4}}=c_{\mathbf{N}_{4}}-\hat{\lambda}_{\mathbf{B}}^{T} \mathbf{a}_{\mathbf{N}_{4}}=4-(1,0)
\begin{pmatrix}
-2\\
3
\end{pmatrix}
=6
$$
\item Passo 2: Teste de otimalidade
\begin{itemize}
\item 2.1: Cálculo da solução básica primal
\[
\underbrace{
\begin{pmatrix}
1&0\\
-1&-1
\end{pmatrix}}_{\mathbf{B}}
\underbrace{
\begin{pmatrix}
\hat{x}_{\mathbf{B}_1} (x_1)\\
\hat{x}_{\mathbf{B}_2} (x_6)
\end{pmatrix}}_{\mathbf{\hat{x}_\mathbf{B}}}
=
\underbrace{
\begin{pmatrix}
2\\
1
\end{pmatrix}}_{\mathbf{b}}
\]
\[
\Rightarrow \mathbf{\hat{x}_\mathbf{B}} = \mathbf{B} \setminus \mathbf{b} =
\begin{pmatrix}
2\\
-3
\end{pmatrix}
\ngeq 0
\]
\item 2.2: Determinar variável a sair da base\\
\[
\hat{x}_{\mathbf{B}_l}=min\{ \hat{x}_{\mathbf{B}_i}, i=1,2\}=min\{2, -3\}=-3
\]
Como $\hat{x}_\mathbf{B_l}<0$, o método prossegue, e $\hat{x}_\mathbf{B_2}$ sairá da base.
		\end{itemize}
		\item Passo 3: Cálculo da direção dual simplex
\[
\underbrace{
\begin{pmatrix}
1&-1\\
0&-1
\end{pmatrix}}_{\mathbf{B}^T}
\mathbf{\eta}_2
=
\underbrace{
\begin{pmatrix}
0\\
-1
\end{pmatrix}}_{-\mathbf{e}_2}
\]
\[
\Rightarrow \mathbf{\eta}_2 = \mathbf{B} \setminus \mathbf{e}_2 =
\begin{pmatrix}
1\\
1
\end{pmatrix}
\]
		\item Passo 4: Determinar passo e variável a entrar na base
\begin{align*}
\widehat{\delta}&=\frac{\widehat{c}_{\mathbf{N}_k}}{\eta_\ell^T \mathbf{a}_{\mathbf{N}_k}}
=\min_{j=1,...,n-m}\left\{\frac{\widehat{c}_{\mathbf{N}_j}}{\eta_2^T \mathbf{a}_{\mathbf{N}_j}} \text { tal que } \eta_2^T \mathbf{a}_{\mathbf{N}_j}>0\right\}\\
&=\min \left\{
\frac{\widehat{c}_{\mathbf{N}_2}}{\eta_{2}^T \mathbf{a}_{\mathbf{N}_2}}, \frac{\widehat{c}_{\mathbf{N}_4}}{\eta_{2}^T \mathbf{a}_{\mathbf{N}_4}}
\right\}\\
&=\min \left\{
\frac{9}{5}, \frac{6}{1}
\right\}=\frac{9}{5}
\end{align*}
Logo $x_{\mathbf{N}_2}$ ($x_2$) entrará na base.
		\item Passo 5: Nova partição básica

Troca-se a segunda coluna de B pela segunda coluna de N.
	\end{itemize}
	\item Terceira iteração
	\begin{itemize}
\item Passo 1: Cálculo da solução básica dual e custos relativos
$$
\begin{gathered}
\underbrace{\left(\begin{array}{cc}
1 & -1 \\
3 & 2
\end{array}\right)}_{\mathbf{B}^{T}} \underbrace{\left(\begin{array}{l}
\hat{\lambda}_{B_{1}}\left(\lambda_{1}\right) \\
\hat{\lambda}_{B_{2}}\left(\lambda_{2}\right)
\end{array}\right)}_{\hat{\lambda}_{\mathbf{B}}}=\underbrace{\left(\begin{array}{l}
1 \\
12
\end{array}\right)}_{\mathbf{c}_{\mathbf{B}}} \\
\Rightarrow \hat{\lambda}_{\mathbf{B}}=B^{T} \backslash \mathbf{c}_{\mathbf{B}}=\left(\begin{array}{l}
\frac{14}{5} \\
\frac{9}{5}
\end{array}\right) \geq 0
\end{gathered}
$$
Solução factível; cálculo dos custos relaticos: 

Associado a $x_{5}$ :
$$
\hat{c}_{\mathbf{N}_{1}}=c_{\mathbf{N}_{1}}-\hat{\lambda}_{\mathbf{B}}^{T} \mathbf{a}_{\mathbf{N}_{1}}=0-\left(\frac{14}{5},\frac{9}{5}\right)
\begin{pmatrix}
-1\\
0
\end{pmatrix}
=\frac{14}{5}
$$
Associado a $x_{2}$ :
$$
\hat{c}_{\mathbf{N}_{2}}=c_{\mathbf{N}_{2}}-\hat{\lambda}_{\mathbf{B}}^{T} \mathbf{a}_{\mathbf{N}_{2}}=0-\left(\frac{14}{5},\frac{9}{5}\right)
\begin{pmatrix}
0\\
-1
\end{pmatrix}
=\frac{9}{5}
$$
Associado a $x_{3}$ :
$$
\hat{c}_{\mathbf{N}_{3}}=c_{\mathbf{N}_{3}}-\hat{\lambda}_{\mathbf{B}}^{T} \mathbf{a}_{\mathbf{N}_{3}}=3-\left(\frac{14}{5},\frac{9}{5}\right)
\begin{pmatrix}
-2\\
-3
\end{pmatrix}
=14
$$
Associado a $x_{4}$ :
$$
\hat{c}_{\mathbf{N}_{4}}=c_{\mathbf{N}_{4}}-\hat{\lambda}_{\mathbf{B}}^{T} \mathbf{a}_{\mathbf{N}_{4}}=4-\left(\frac{14}{5},\frac{9}{5}\right)
\begin{pmatrix}
-2\\
3
\end{pmatrix}
=\frac{21}{5}
$$
\item Passo 2: Teste de otimalidade
\begin{itemize}
\item 2.1: Cálculo da solução básica primal
\[
\underbrace{
\begin{pmatrix}
1&3\\
-1&2
\end{pmatrix}}_{\mathbf{B}}
\underbrace{
\begin{pmatrix}
\hat{x}_{\mathbf{B}_1} (x_1)\\
\hat{x}_{\mathbf{B}_2} (x_2)
\end{pmatrix}}_{\mathbf{\hat{x}_\mathbf{B}}}
=
\underbrace{
\begin{pmatrix}
2\\
1
\end{pmatrix}}_{\mathbf{b}}
\]
\[
\Rightarrow \mathbf{\hat{x}_\mathbf{B}} = \mathbf{B} \setminus \mathbf{b} =
\begin{pmatrix}
\frac{1}{5}\\
\frac{3}{5}
\end{pmatrix}
\geq 0
\]
\item 2.2: Determinar variável a sair da base\\
\[
\hat{x}_{\mathbf{B}_l}=min\{ \hat{x}_{\mathbf{B}_i}, i=1,2\}=min\left\{\frac{1}{5}, \frac{3}{5}\right\}=\frac{1}{5}
\]
Como $\hat{x}_\mathbf{B_l}\geq0$, o método para.
		\end{itemize}
		
	\end{itemize}
	\end{itemize}	
\section{Análise de solução}
A solução ótima do PL primal é $\mathbf{x*}=\left(\frac{1}{5}, \frac{3}{5}\right)^T$, e o valor ótimo da função do primal é $f(\mathbf{x*})=\frac{37}{5}$, e, como visto, $\lambda*=\left(\frac{14}{5}, \frac{9}{5}\right)^T$, logo o valor ótimo da função do dual é $g(\lambda*)=\frac{37}{5}$, o que coincide com o do primal e valida o resultado. Vale ressaltar que $\left(\frac{14}{5}, \frac{9}{5}\right)^T$ é o vetor multiplciador simplex na solução do primal, e como este é não negativo, caso o método usado fosse o primal simplex e ele se encontrasse nesse ponto, de fato não haveria direção a se tomar nessa iteração e o método pararia.
\end{document}
