\section{Experimento}
Com o intuito de atingir os objetivos propostos no roteiro experimental, foi divisado o seguinte procedimento:
\begin{enumerate}
    \item Sobre um papel milimetrado sobre o fundo de uma cuba de plástico de fundo plano cheia com alguns milímetros de solução aquosa de $\text{Cu}_2\text{SO}_4$, dois eletrodos de cerca de 30 cm são posicionados paralelos entre si com uma distância de 14 cm. As barras são posicionadas de tal forma que suas extremidades estão alinhadas entre si e com um espaçamento igual após o término da região milimetrada de ambos os lados, por motivos de simetria;
    \item A fonte é conectada aos eletrodos, seguindo o esquema fornecido, ligada e ajustada para fornecer uma voltagem de 2.5 V e uma corrente de mais de 0.03 A (mais solução a ser colocada em caso da corrente ser menor que tal valor), no caso a corrente registrada foi de 0.06 A a 0.08 A;
    \item O voltímetro é ligado também seguindo o esquema apresentado, com sua entrada terra conectada ao eletrodo carregado negativamente e a outra a uma ponteira;
    \item Um plano cartesiano cuja origem se encontra na primeira intersecção do papel milimetrado na extremidade esquerda do eletrodo carregado positivamente é definido (como pode ser visto no esquema apresentado na Fig.~\ref{fig:1});
    \item São escolhidos 4 pontos e seus simétricos (com relação ao eixo de simetria paralelo ao eixo $y$ definido, com $x=14$ cm) para fornecer variações significativas nos dois eixos cartesianos. Estes pontos são testados com a ponteira do voltímetro, tomando-se o máximo de cuidado para garantir a perpendicularidade dessa última com relação ao plano e com o responsável tomando um ponto de vista superior para melhor posicioná-la no papel milimetrado, evitando paralaxe. Os valores são anotados e utilizados para garantir a simetria experimental, que garantirá a possibilidade de testar metade da área milimetrada entre os eletrodos, posteriormente extrapolando o resultado para o outro lado do eixo de simetria. Vale ressaltar ainda que, durante todas as medições, a ponteira foi utilizada por um mesmo membro do grupo, a fim de evitar erros e incertezas inerentes a quem realizou a medição que poderiam interferir nos dados;
    \item São medidos os pontos ao longo do eixo de simetria acima descrito, com espaçamento de 0.5 cm entre os pontos;
    \item São medidos os pontos na matriz de pontos com espaçamento de 2 cm em ambos os eixos, começando da origem do plano definido;
    \item A ponteira é posicionada no centro do eletrodo positivamente carregado e são repetidos os 2 procedimentos acima;
    \item O aro é posicionado no centro entre os eletrodos (centro do aro nas coordenadas  $(x, y)=(14, 7)$) com a ajuda de um paquímetro para obter-se o diâmetro do aro. Novamente são repetidos os 2 procedimentos anteriores.
\end{enumerate}

As variáveis dependentes do experimento são o potencial elétrico, que depende da posição da medida no plano, da inclinação da ponteira em relação ao plano (cuja influencia será discutida mais a frente), da tensão da fonte e da distância entre os eletrodos, e consequentemente o campo elétrico, que depende do potencial segundo a equação apresentada no roteiro. As variáveis independentes são aquelas mencionadas na dependência do potencial. Outra variável independente cujo o efeito não pôde ser explorado é a concentração da solução aquosa e sua homogeneidade, que foram assumidos como satisfatórios para a execução do experimento.

A partir dos 4 pontos e seus simétricos descritos no item 5 acima, foi averiguado que existe uma simetria na distribuição do potencial, pois as diferenças encontradas estavam presentes na segunda casa decimal (com incerteza na leitura do voltímetro sendo da ordem da terceira casa decimal).

Para calcular o campo elétrico nas 3 situações no centro das placas, na frente da ponteira e no centro do aro, foi usada a Eq. (1) do roteiro experimental. Os pontos usados variam na direção do eixo $y$ ($(14,7)$ é o ponto inicial e $(14,7.5)$ o final), pois, pelos gráficos das curvas de nível plotados, essa é a direção perpendicular aproximada de tais curvas. Os valores obtidos podem ser vistos na Tab.~\ref{tab:1}, cujas incertezas foram obtidas na seção de incertezas. Observando os valores obtidos para o campo elétrico no ponto mencionado, é possível notar que, de fato, o campo elétrico possui módulo maior na configuração 2(b) do que na 2(a) devido ao efeito de ponta, enquanto na configuração 2(c), na área interior do aro, é possível notar que o módulo do campo, apesar de não ser nulo, é próximo de 0, como o esperado.

Analisando o gráfico dos potenciais da configuração da Figura 2(a) (Fig.~\ref{fig:2}) pode se identificar o alargamento das distâncias entre as faixas equipotenciais próximas às bordas, caracterizando o efeito de borda consequente do fato dos eletrodos não serem idealmente infinitos. No entanto, pode se destacar o paralelismo entre as curvas de potencial quando analisadas próximas à faixa central do eixo $x$. A respeito do aspecto gráfico, diz-se que o gráfico não sofreu “perturbação” devido à objetos que acarretassem um fenômeno de potencial significativa (pontas e gaiolas de Faraday, por exemplo).

Comparando a Fig.~\ref{fig:3} com a Fig.~\ref{fig:2}, que representa o caso com a  ponteira, é possível identificar um novo arranjo das linhas equipotenciais. Esta nova  configuração mostra uma distorção considerável próxima à ponta, ocasionada graças ao efeito das pontas, no qual as cargas tendem a distribuir-se de modo a haver um acúmulo maior nas extremidades pontiagudas. Tal efeito faz com que as linhas equipotenciais se contorçam em torno da ponta de forma acentuada próximo a ela e de forma quase imperceptível (comparável ao caso 2(a)) longe. 

Na Fig.~\ref{fig:4} da configuração 2(c), é possível ver a redistribuição das linhas equipotenciais com a inserção do aro. Estas foram afastadas do aro, criando uma espécie de circunferência, e formando uma região equipotencial (cerca de 1,23 V) limitada pelo mesmo, como é característico do efeito da gaiola de Faraday que era visado.

Analisando a Fig.~\ref{fig:8}, na curva referente ao caso 2(a), é possível averiguar que os valores no eixo de simetria variam de acordo com a distância que se encontram da placa superior e inferior de forma linear, isto é, os valores do potencial decaem do potencial próximo ao da fonte (aproximadamente 2.5 V, no eletrodo positivamente carregado) de forma quase constante até potencial nulo no eletrodo negativo. A inclinação levemente diferente no começo do eixo de simetria pode ser resultado do efeito de bordas. Já no caso 2(b), valores do potencial em cima da ponta são constantes e decrescem ao passo em que se distanciam da extremidade da ponta de forma quase constante, semelhante ao primeiro caso. A inclinação mais acentuada logo após a ponta pode ser indicativo de um comportamento não linear da distribuição de carga nessa região. No caso 2(c), partindo da placa positivamente carregada, o valores variam de forma constante, até atingirem um platô na região dentro do aro, sem apresentar qualquer variação significativa, logo retornando a decair ao sair dessa região, até o valor nulo.