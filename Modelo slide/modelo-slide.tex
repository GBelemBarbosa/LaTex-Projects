\documentclass[11pt]{beamer}
\usepackage[makeroom]{cancel}
\usepackage{verbatim}

\usepackage{bigints}

%\usepackage{unicode-math}
%\usepackage{mathtools}


\usepackage{listings}

%\usepackage{cmbright,fourier}

\newenvironment{metaverbatim}{\verbatim}{\endverbatim}

\usepackage{ulem}
\usepackage{mathtools}
%-=-=-=-=-=-=-=-=-=-=-=-=-=-=-=-=-=-=-=-=-=-=-=-=
%        LOADING PACKAGES
%-=-=-=-=-=-=-=-=-=-=-=-=-=-=-=-=-=-=-=-=-=-=-=-=
%\usepackage[utf8]{inputenc}
\usepackage{tikz}
\usepackage{amsmath}
\usepackage{amssymb}
\usepackage{amssymb}
\usetikzlibrary{arrows}



\usepackage{chronology}
\usepackage{multicol}
\usepackage{url}

              

\title{MA311 - Cálculo III}
\author{{Ricardo M Martins}\\ rmiranda@unicamp.br\\ 
\vspace{0.3cm}\url{http://www.ime.unicamp.br/~rmiranda}
}

\usepackage{amsmath}
\usepackage{amsthm}
\usepackage{amssymb}
\usepackage{amscd}
\usepackage{amsfonts}
\usepackage{amsbsy}
\usepackage{pdfpages}

\renewcommand\rmdefault{hpv}

%\usepackage{mathspec}
%\setmainfont{Gotham Book}
%\setmathrm{Gotham Book}
%\setmathfont(Digits,Latin){Gotham Book}


\newtheorem{lema}{Lema}
\BeforeBeginEnvironment{lema}{%
    \setbeamercolor{block title}{fg=white,bg=blue!50!black}
    \setbeamercolor{block body}{fg=black, bg=blue!10!white}
}

\newtheorem{corolario}{Corolário}
\BeforeBeginEnvironment{corolario}{%
    \setbeamercolor{block title}{fg=white,bg=orange!70!white}
    \setbeamercolor{block body}{fg=black, bg=orange!10!white}
}

\newtheorem{teorema}{Teorema}
\BeforeBeginEnvironment{teorema}{%
    \setbeamercolor{block title}{fg=white,bg=green!30!black}
    \setbeamercolor{block body}{fg=black, bg=green!20!white}
}


\newtheorem{exemplo}{Exemplo}
\BeforeBeginEnvironment{exercicio}{%
    \setbeamercolor{block title}{fg=white,bg=orange!85!black}
    \setbeamercolor{block body}{fg=black, bg=orange!20!white}
}


\newtheorem{exercicio}{Exercício}
\BeforeBeginEnvironment{exemplo}{%
    \setbeamercolor{block title}{fg=white,bg=red!70!black}
    \setbeamercolor{block body}{fg=black, bg=red!20!white}
}

\newtheorem{exerc}{Exercício}
\BeforeBeginEnvironment{exemplo}{%
    \setbeamercolor{block title}{fg=white,bg=red!70!black}
    \setbeamercolor{block body}{fg=black, bg=red!20!white}
}



\newtheorem{prova}{Prova}
\BeforeBeginEnvironment{prova}{%
    \setbeamercolor{block title}{fg=white,bg=red!70!black}
    \setbeamercolor{block body}{fg=black, bg=red!20!white}
}


\newtheorem{solucao}{Solução}
\BeforeBeginEnvironment{solucao}{%
    \setbeamercolor{block title}{fg=black,bg=yellow!85!black}
    \setbeamercolor{block body}{fg=black, bg=yellow!20!white}
}

\newtheorem{sol}{Solução}
\BeforeBeginEnvironment{solucao}{%
    \setbeamercolor{block title}{fg=black,bg=yellow!85!black}
    \setbeamercolor{block body}{fg=black, bg=yellow!20!white}
}

\def\C{\mathbb C}
\def\nx{\mathbf{x}}
\def\Z{\mathbb Z}
\def\N{\mathbb N}
\def\Q{\mathbb Q}
\def\H{\mathbb H}
\def\F{{\mathbb F}_q}
\def\cc{{\mathcal C}}
\def\bv{{\bf v}}
\def\rr{{\mathcal R}_n}
\def\im{{\bf i}}
\def\supp{\mathop{{\rm supp}}}
\def\zen{\mathop{\mathcal{Z}}}
\def\base{\mathcal{B}}
\def\Geny{\mathcal{G}}
\def\vezes{\mathop{{\rm times}}}
\def\Gal{\mathop{{\rm Gal}}}
\def\alge{\mathcal{A}}
\def\pr{\noindent{\sc proof. }}
\def\wh{\widehat}

%\def\lhd{\vartriangleleft}
%\newcommand{\fim}{\hspace*{\fill} $\square$\medskip}
%\newcommand{\ndv}{ \ {\mid \kern -0.70 em {\scriptstyle \not}} \ \ }

\newcommand{\rn}[1]{\mathbb{R}^{#1}}
\newcommand{\xpt}[1]{\dot{#1}}
\newcommand{\en}[1]{\textbf{e}_#1}
\newcommand{\ex}[1]{\noindent {\bf #1.}}
\newcommand{\zz}{\mathbb{Z}_{2}}

\newcommand{\bq}{\begin{equation}}
\newcommand{\e}{\varepsilon}
\newcommand{\T}{\theta}
\newcommand{\f}{\phi}
\newcommand{\eq}{\end{equation}}



\DeclareMathOperator{\sen}{sen}
\newcommand{\senx}{\sen(x)}

\newcommand{\bx}{{\bf x}}

\newcommand{\tr}{{\rm tr}}

\newcommand{\R}{\mathbb R}

\newcommand{\ra}{\rightarrow}

\newcommand{\Rm}{\mathbb R^m}
\newcommand{\Rn}{\mathbb R^n}
\newcommand{\RR}{\mathbb R^2}
\newcommand{\RRR}{\mathbb R^3}



\usetheme{Boadilla}

\begin{document}

\begin{frame}{}
\maketitle
\end{frame}

%%%%%%%%%%%%%%%%%%%%%%

\begin{frame}[t]{Integrais}

Sabemos como encontrar funções $y(t)$ {\it boas} e com a propriedade de que
\begin{equation}\label{0}\dfrac{dy(t)}{dt}=f(t),\end{equation} onde $f(t)$ é outra função: \pause fixando $y(0)=c$ então \[y(t)=\int_0^t f(s)\,ds+c.\] \pause 

Estaremos interessados agora em equações parecidas com \eqref{0}, mas com o lado direito também dependendo de $y(t)$.


\end{frame}




\begin{frame}{Um primeiro exemplo}

% https://en.wikipedia.org/wiki/Van_der_Pol_oscillator
\begin{block}{Pergunta 1}
Você consegue achar uma função $u(t)$ que satisfaz 
\begin{equation}\label{1}
\dfrac{d^2u}{dt^2}(t)+u(t)=0?
\end{equation}
\end{block}
%\textcolor{blue}{Existe alguma função $u(t)$ periódica de período $T>0$  que resolve a equação?}

\end{frame}






\begin{frame}[t]{Colocando o problema de outra forma}

Vamos obter equações equivalentes a $u''(t)+u(t)=0$. \pause 

\begin{itemize}
\item Seja $u(t)$ uma solução da equação \eqref{1}. \pause 
\item Denote $v(t)=u'(t)$. \pause 
\item Então $v'(t)=u''(t)$. \pause 
\item Como $u(t)$ é solução de \eqref{1}, $u''(t)=-u(t)$. \pause 
\item Logo $v'(t)=u''(t)=-u(t)$. \pause 
\end{itemize}

O sistema abaixo é equivalente a \eqref{1}:
\begin{equation}\label{2}\left\{
\begin{array}{lcl}
u'(t)&=&v(t),\\
v'(t)&=&-u(t).
\end{array}
\right.
\end{equation}

\end{frame}

\begin{frame}[t]{Colocando o problema de outra forma}

\[
\left\{\begin{array}{lcl}
u'(t)&=&v(t),\\
v'(t)&=&-u(t).
\end{array}
\right.
\]
 \pause 
\begin{itemize}
\item Antes procurávamos uma função \textcolor{red}{(mas que dependia de uma derivada de 2a ordem).}
\item Agora procuramos duas funções \textcolor{red}{(mas que dependem somente de derivadas de 1a ordem).}
\end{itemize}
 \pause 
Quais as soluções do sistema de equações acima?\pause 
\[
\left\{\begin{array}{lcl}
u(t)&=&a\cos(t)+b\sen(t),\\
v(t)&=&-a\sen(t)+b\cos(t), \ a,b\in\R.
\end{array}\right.
\]


\end{frame}

\begin{frame}[t]{Colocando o problema de outra forma}

Considere a curva parametrizada
\[
\begin{array}{lcl}
\vspace{0.3cm}\alpha(t)&=&\big(u(t),v(t)\big)\\
&=&\big(\textcolor{blue}{a\cos(t)+b\sen(t)},\textcolor{red}{-a\sen(t)+b\cos(t)} \big), \ t\in\R.
\end{array}
\]

Qual o traço de $\alpha$? \pause 

Note que
\[\big(\textcolor{blue}{a\cos(t)+b\sen(t)}\big)^2+\big(\textcolor{red}{-a\sen(t)+b\cos(t)}\big)^2=a^2+b^2,\]logo o traço de $\alpha$ está contido numa circunferência de raio $a^2+b^2$ (na verdade, é a circunferência).
\end{frame}


\begin{frame}[t]{Colocando o problema de outra forma}

\begin{itemize}
\item Todas as circunferências de raio $r\geq 0$ são soluções de \eqref{2}. \pause 
\item Dado um ponto $(p,q)\in\RR$, existe uma única circunferência que é solução de \eqref{2} e  passa por $(p,q)$. \pause 
\end{itemize}

\begin{teorema}[Existência e Unicidade] Seja $(p,q)\in\RR$ e $F,G$ funções boas. Existe uma \textcolor{red}{única} solução da equação
\[
\left\{\begin{array}{lcl}
u'(t)&=&F(u(t),v(t)),\\
v'(t)&=&G(u(t),v(t))
\end{array}
\right.
\]
que satisfaz $u(0)=p$ e $v(0)=q$.
\end{teorema}

\textcolor{magenta}{Este teorema é muito mais geral do que isto.}
\end{frame}

\begin{frame}{Colocando o problema de outra forma}

\textcolor{magenta}{Será que todas as soluções não-constantes de \eqref{1} são periódicas?}
\pause 
\begin{center}
\textcolor{blue}{Sim!}
\end{center}

\end{frame}




\begin{frame}[t]{E a geometria?}



$\left\{\begin{array}{lcl}
u'(t)&=&v(t),\\
v'(t)&=&-u(t).
\end{array}
\right. \ \textcolor{white}{{\footnotesize \textcolor{white}{\leftarrow}\text{E se pensarmos nisto como um campo vetorial?}}}
$

\[{\footnotesize \alpha(t)=\big(a\cos(t)+b\sen(t),-a\sen(t)+b\cos(t) \big)}\]



\end{frame}










\begin{frame}[t]{E a geometria?}



$\left\{\begin{array}{lcl}
u'(t)&=&v(t),\\
v'(t)&=&-u(t).
\end{array}
\right. \ \textcolor{red}{{\footnotesize \leftarrow\text{E se pensarmos nisto como um campo vetorial?}}}
$

\[{\footnotesize \alpha(t)=\big(a\cos(t)+b\sen(t),-a\sen(t)+b\cos(t) \big)}\]




\end{frame}



\begin{frame}[t]{E a geometria?}



$\left\{\begin{array}{lcl}
u'(t)&=&v(t),\\
v'(t)&=&-u(t).
\end{array}
\right. \ \textcolor{blue}{{\footnotesize \leftarrow\text{Teremos um campo vetorial tangente às soluções!}}}
$

\[{\footnotesize \alpha(t)=\big(a\cos(t)+b\sen(t),-a\sen(t)+b\cos(t) \big)}\]



\end{frame}


%
%
%\begin{frame}[t]{E a geometria?}
%
%Seja $X(u,v)=(v,-u)$ um campo vetorial. Qual a relação deste campo com as soluções de
%\[\left\{\begin{array}{lcl}
%u'(t)&=&v(t),\\
%v'(t)&=&-u(t)?
%\end{array}
%\right.\]
%
%Este campo vetorial é sempre tangente às soluções!
%
%\end{frame}
%
%
%\begin{frame}[t]{E a geometria?}
%
%
%\begin{center}\includegraphics[width=0.7\textwidth]{aula1-fig4.pdf}
%\end{center}
%
%
%\end{frame}
%
%



\begin{frame}[t]{E a geometria?}

\begin{block}{Resumindo}
Supondo boas funções $P,Q$, o campo vetorial \[X(x,y)=\big(P(x,y),Q(x,y)\big)\] é tangente às soluções do sistema de equações diferenciais
\[\left\{\begin{array}{lcl}
x'(t)&=&P\big(x(t),y(t)\big),\\
y'(t)&=&Q\big(x(t),y(t)\big),
\end{array}
\right.\]
e as soluções deste sistema existem e são únicas (fixado um ponto $(p,q)$ por onde a solução passa).
\end{block}

\textcolor{magenta}{Uma solução de um sistema como o acima é chamada de órbita ou trajetória do sistema.}

\end{frame}

%
%\begin{frame}[t]{Segundo exemplo}
%
%% https://en.wikipedia.org/wiki/Van_der_Pol_oscillator
%\begin{block}{Pergunta}
%Você consegue achar uma função periódica $u(t)$ que satisfaz 
%\[
%u''(t)\textcolor{red}{-}u(t)=0?
%\]
%\end{block}
%
%Usando nosso truque, o desafio é equivalente a encontrar uma solução periódica (curva fechada) do sistema
%\begin{equation}\label{3}\left\{\begin{array}{lcl}
%u'&=&v,\\
%v'&=&u.
%\end{array}
%\right.\end{equation}
%
%Como é este campo vetorial?
%\end{frame}
%
%\end{document}



\begin{frame}[t]{Segundo exemplo}

% https://en.wikipedia.org/wiki/Van_der_Pol_oscillator
\begin{block}{Pergunta 2}
Você consegue achar uma função periódica $u(t)$ que satisfaz 
\[
u''(t)+u(t)=\textcolor{red}{2u'(t)}?
\]
\end{block}
 \pause 
Usando nosso truque, o desafio é equivalente a encontrar uma solução periódica (curva fechada) do sistema
\begin{equation}\label{3}\left\{\begin{array}{lcl}
u'&=&v,\\
v'&=&-u+\textcolor{red}{2v}.
\end{array}
\right.\end{equation}





\end{frame}

%
%\begin{frame}[t]{Segundo exemplo}
%\[\left\{\begin{array}{lcl}
%u'&=&v,\\
%v'&=&-u+\textcolor{red}{2v}.
%\end{array}
%\right.\]
%\pause 
%\textcolor{magenta}{Suponha que exista uma solução suave $\alpha(t)=\big(u(t),v(t)\big)$ $T$-periódica para o sistema \eqref{3}.} 
%
%\pause 
%
%\begin{itemize}
%\item  $\displaystyle \oint_\alpha ()\displaystyle 0=\oint_\alpha v\,dv-(2v-u)du$ \pause 
%\item Green: $\displaystyle \oint_\alpha v\,dv-(2v-u)du=\iint_R 2\,du\,dv$ \pause 
%\item Logo: $\displaystyle \iint_R 2\,du\,dv =0$, absurdo! \pause 
%\end{itemize}
%
%\textcolor{red}{Portanto, não temos nenhuma solução periódica!} Como são as soluções?
%\end{frame}
%



\begin{frame}[t]{Segundo exemplo}

Como entramos uma solução do sistema anterior?

\begin{itemize} \pause 
\item Propondo soluções $u(t),v(t)$ polinomiais: \textcolor{red}{não funciona.} \pause 
\item Soluções $u(t),v(t)$ trigonométricas: \textcolor{red}{não funciona.} \pause 
\item Misturando funções polinomiais, exponenciais e trigonométricas: \textcolor{blue}{funciona!} \pause 
\end{itemize}


Propondo uma solução da forma $u(t)=e^{a_1t}(a_2+a_3t)$ e $v(t)=e^{b_1t}(b_2+b_3t)$, encontraremos que
\[u(t)= e^t(a_2+a_3t), \ v(t)=e^t(a_2+a_3+a_3t),\] para todo $a_2,a_3\in\R$.
\end{frame}


\begin{frame}[t]{Encontrando as soluções}


Note que se calcularmos $||(u(t),v(t))||$ vamos perceber que $$||(u(t),v(t))||\ra \infty,$$ logo nenhuma solução pode ser periódica.


\end{frame}


\begin{frame}[t]{Encontrando as soluções}

\textcolor{magenta}{Importante:} originalmente, durante a aula, o argumento que garantia a inexistência da solução periódica foi baseado no Teorema de Green.

O argumento estava correto, mas faltava explicar o motivo de uma das integrais dar zero.

Aqui estão mais alguns detalhes.


Suponha que exista uma solução suave $\alpha(t)=\big(u(t),v(t)\big)$ $T$-periódica para o sistema \eqref{3}. Seja $R$ a região delimitada pela curva $\alpha$.

\end{frame}

\begin{frame}[t]{Encontrando as soluções}

Então
\[\oint_{\alpha} (v,-u+2v)\cdot n\,ds=\oint_{\alpha} v\,dv-(2v-u)\,du,\]
mas a integral da esquerda precisa ser zero, já que o campo $(v,-u+2v)$ é tangente à curva $\alpha$ e $n$, sendo normal a $\alpha$, será normal também a $(v,-u+2v)$.

Por outro lado, pelo Teorema de Green,
\[\displaystyle \oint_\alpha v\,dv-(2v-u)du=\iint_R 2\,du\,dv.\] e certamente esta integral não pode dar zero (pois é uma área). \textcolor{magenta}{De onde vem o absurdo? De termos suposto que existe a curva $\alpha$, solução periódica.}




%\begin{itemize}
%\item  $\displaystyle \oint_\alpha ()\displaystyle 0=\oint_\alpha v\,dv-(2v-u)du$ \pause 
%\item Green: $\displaystyle \oint_\alpha v\,dv-(2v-u)du=\iint_R 2\,du\,dv$ \pause 
%\item Logo: $\displaystyle \iint_R 2\,du\,dv =0$, absurdo! \pause 
%\end{itemize}


\end{frame}


%
%\begin{frame}[t]{ \ }
%\begin{center}
%\includegraphics[width=0.6\textwidth]{aula1-fig2.pdf}
%\end{center}
%\[\textcolor{blue}{u(t)=ae^{t/2}\cos\bigg(\dfrac{\sqrt{3}}{2}t\bigg)+be^{t/2}\sen\bigg(\dfrac{\sqrt{3}}{2}t\bigg)}\]
%
%\textcolor{magenta}{Nenhuma solução é periódica!} Como perceber isto sem conhecer explicitamente a solução?
%
%
%\end{frame}
%
%
%


%%%
%\begin{frame}[t]{ \ }
%% https://en.wikipedia.org/wiki/Van_der_Pol_oscillator
%Considere o sistema
%\[
%\left\{\begin{array}{lcl}
%\vspace{0.2cm}x'(t)&=&y(t),\\
%y'(t)&=&-x(t)\textcolor{black}{,}\textcolor{white}{+\mu \cdot \big(1-x(t)^2\big)\cdot y(t),}
%\end{array}
%\right.
%\]
%onde $x(t),y(t)$ são funções de $t$ diferenciáveis.\\
%
%\pause
%
%\medskip
%
%\textcolor{blue}{Existem funções $x(t),y(t)$ não-constantes e periódicas de período $T>0$  que sa\-tis\-fazem o sistema?}
%
%\pause 
%
%\textcolor{magenta}{Todas as soluções não-constantes são periódicas?}
%\end{frame}



\begin{frame}[t]{Terceiro exemplo}
% https://en.wikipedia.org/wiki/Van_der_Pol_oscillator

%%%%
\begin{block}{Pergunta 3}
Você consegue achar uma função periódica $u(t)$ que satisfaz 
\[
u''(t)-\textcolor{red}{(1-u^2)u'}+u=0?
\]
\end{block}

ou:
\begin{block}{Pergunta 3'}
Você consegue achar uma solução periódica (curva fechada) do sistema abaixo?
\begin{equation}\label{5}\left\{\begin{array}{lcl}
u'&=&v,\\
v'&=&-u+\textcolor{red}{(1-u^2)v}.
\end{array}
\right.\end{equation}
\end{block}

%%%%
%
%\begin{block}{Pergunta 3}
%Você consegue achar uma função periódica $u(t)$ que satisfaz \[u''(t)-(1-u^2)u'+u=0?\]
%\end{block}
%
%
%Considere o sistema
%\[
%\left\{\begin{array}{lcl}
%\vspace{0.2cm}x'(t)&=&y(t),\\
%y'(t)&=&-x(t)\textcolor{white}{,}\textcolor{red}{+\mu \cdot \big(1-x(t)^2\big)\cdot y(t),}
%\end{array}
%\right.
%\]
%onde $x(t),y(t)$ são funções de $t$ diferenciáveis.\\
%
%
%\medskip
%
%\textcolor{blue}{Existem funções $x(t),y(t)$ não-constantes e periódicas de período $T>0$  que sa\-tis\-fazem o sistema? Suponha  $\mu$ um parâmetro pequeno.}
%
%\textcolor{magenta}{xx}

\end{frame}

\begin{frame}[t]{Problemas, muitos problemas}

\[\left\{\begin{array}{lcl}
u'&=&v,\\
v'&=&-u+\textcolor{red}{(1-u^2)v}.
\end{array}
\right.
\]

Esta equação é conhecida com equação de {\it van der Pol}. \pause 

\textcolor{blue}{Podemos provar que existe uma {\bf única} solução periódica, mas não conseguimos exibir uma parametrização para a solução.}
 \pause 

Ou seja: não podemos obter a solução de forma explícita.
 \pause 

\textcolor{magenta}{É aí que entra a teoria qualitativa das equações diferenciais: obter informações sobre as soluções da equação diferencial somente a partir do campo vetorial.}

\end{frame}

\begin{frame}[t]{Problemas, muitos problemas}

\begin{minipage}[t]{0.45\textwidth}
        \vspace{0pt}
\end{minipage}%
    \hfill
    \begin{minipage}[t]{0.45\textwidth}
        \vspace{0pt}
A curva vermelha é uma aproximação para a solução periódica de van der Pol.\\ \pause 

As soluções ``internas" se aproximam da curva vermelha, e o mesmo acontece com as soluções ``externas".\\ \pause 

\textcolor{magenta}{A curva vermelha é um típico {\it ciclo limite atrator}.}
\end{minipage}
\end{frame}

\begin{frame}[t]{Nosso curso}

Discutiremos os seguintes problemas neste curso: \pause 

\begin{itemize}
\item Existência de soluções para EDOs \pause 
\item Soluções de equações de primeira e segunda ordem \pause 
\item Transformações de Laplace e Lagrange \pause 
\item Soluções em série \pause 
\item Rápida introdução às EDPs \pause 
\item Sistemas lineares \pause 
\item Introdução aos sistemas dinâmicos \pause 
\item Modelos matemáticos
\end{itemize}

\end{frame}


\begin{frame}[t]{Nosso curso}

Referências básicas:
\begin{itemize}
\item William E. Boyce, Richard C. DiPrima, {\it Elementary differential equations and boundary value problems}.
\item Djairo Guedes Figueiredo \& Aloisio Freiria Neves, {\it Equações diferenciais aplicadas}. 
\item Gilbert Strang, {\it Differential equations and linear algebra}.
\item George F. Simmons, {\it Differential equations with applications and historical notes}. 
\item Morris W. Hirsch and Stephen Smale, {\it Differential equations, dynamical systems and linear algebra}. 
\end{itemize}


\end{frame}

%%

%%

\begin{frame}[t]{Nosso curso}

Critérios de avaliação \pause 

\begin{itemize}
\item peso 1: Atividades online (Classroom) \pause 
\item peso 2: Testinhos (aula do PED) \pause 
\item peso 3: Prova 1 24/04 \pause 
\item peso 4: Prova 2 29/06 \pause 
\item peso 0, mas vale Bis: competições no Kahoot!. \pause 
\end{itemize}

Cálculo da média final: média ponderada. \pause 

Exame final: 17/07
\end{frame}


\begin{frame}[t]{Nosso curso}

O que é esperado que vocês saibam? \pause 

\begin{itemize}
\item Derivar e integrar com bastante habilidade. \pause 
\item Ter alguma lembrança sobre Séries de Taylor. \pause 
\item Habilidade para parametrizar curvas e entender o traço de uma parametrização. \pause 
\item Conhecimentos de autovalores/autovetores e diagonalização. \pause 
\item Noções sobre modelagem matemática (pode ser as que você adquiriu nos cursos de física) \pause 
\end{itemize}

\end{frame}


\begin{frame}[t]{Kahoot!}

\begin{center}\url{https://kahoot.it}
\end{center}

\pause 
\textcolor{magenta}{Aula de 6a: testinho de diagnóstico} \pause 

\begin{flushright}Até mais!
\end{flushright}
\end{frame}







\begin{frame}[t]{Algumas coisas burocráticas}

\begin{itemize}
\item Plano de Desenvolvimento/Ementa
\item Provas: 11/04, 23/05 e 27/06 (Exame 10/07)
\item Bibliografia: \textcolor{blue}{Guidorizzi}, \textcolor{blue}{Stewart}, \textcolor{blue}{Apostol}, \textcolor{blue}{Simmons}
\item PEDs/PADs
\item Google Classroom e e-mail oficial
\item WolframAlpha \& Mathematica
\item Participação em sala de aula
\item Média final
\end{itemize}

\end{frame}


\begin{frame}[t]{Trabalho}

\begin{block}{Bloco xxx }
asdfg
\end{block}

\begin{exemplo}Determine o trabalho realizado pelo campo 
\end{exemplo}

\begin{exercicio} kawabanga
\end{exercicio}

\begin{teorema}
asdfg
\end{teorema}

\begin{solucao} kaka
\end{solucao}
\end{frame}





\setbeamercolor{background canvas}{bg=yellow!20}
\begin{frame}[t]

\textcolor{red}{\large Próxima aula}

\begin{itemize}
\item Vamos resolver em grupo a P3 de 2018 (6a/manhã).
\end{itemize}
\end{frame}


\end{document}
